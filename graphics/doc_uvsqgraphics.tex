% !TEX encoding = UTF-8 Unicode
\documentclass{article}

\usepackage[utf8]{inputenc}
\usepackage[french]{babel}
\setlength{\topmargin}{-1.54cm}
\setlength{\oddsidemargin}{-0.04cm}
\setlength{\evensidemargin}{-0.04cm}
\setlength{\textwidth}{17cm}
\setlength{\textheight}{23cm}
\usepackage{graphics}
\usepackage{graphicx}
\usepackage{hyperref}
\hypersetup{
hyperindex=true,    %ajoute des liens dans les index.
colorlinks=true,    %colorise les liens
breaklinks=true,    %permet le retour à la ligne dans les liens trop longs
urlcolor= blue,     %couleur des hyperliens
linkcolor= blue,    %couleur des liens internes
bookmarks=true,     %crée des signets pour Acrobat
}

\usepackage{verbatim}
\usepackage{color}
\usepackage{xcolor}
\usepackage{mdframed}
\definecolor{purple}{rgb}{0.5,0,0.5}
\definecolor{shadecolor}{gray}{0.85}
\newcommand\code[1]{
\begin{mdframed}[linecolor=purple,backgroundcolor=blue!10]
{\tt
#1
}
\end{mdframed}
}

\title{Documentation \texttt{uvsqgraphics}}
\author{Franck.Quessette@uvsq.fr}
\date{Mai 2016}

\begin{document}
\maketitle
\tableofcontents

\section{Types, Variables, Constantes}

\subsection{Types}
Le type \texttt{POINT} permet de stocker les coordonnées d'un point.
\code{
typedef struct point \{int x,y;\} POINT; \\
typedef Uint32 COULEUR; \\
typedef int BOOL;
}

\subsection{Variables}
La largeur et la hauteur de la fenêtre graphique.
\code{
int WIDTH; \\
int HEIGHT; \\
\#define HAUTEUR\_FENETRE HEIGHT\\
\#define LARGEUR\_FENETRE WIDTH
}

\noindent
Les variables \texttt{WIDTH} et \texttt{HEIGHT} sont initialisées lors de l'appel à \texttt{init\_graphics()}.

\subsection{Constantes}

Les 16 couleurs en français et 140 en anglais (voir la liste en annexe).
Quelques couleurs de base en français:
\code{
\#define noir~~~ 0x000000 \\
\#define gris~~~ 0x777777 \\
\#define blanc~~ 0xffffff \\
\#define rouge~~ 0xff0000 \\
\#define vert~~~ 0x00ff00 \\
\#define bleu~~~ 0x0000ff \\
\#define jaune~~ 0x00ffff \\
\#define cyan~~~ 0xffff00 \\
\#define magenta 0xff00ff
}

Les constantes booléennes:
\code{
\#define TRUE 1 \\
\#define True 1 \\
\#define true 1 \\
\#define FALSE 0 \\
\#define False 0 \\
\#define false 0 
}


\section{Affichage}
\subsection{Initialisation de la fenêtre graphique}

\code{
void init\_graphics(int W, int H);
}

\subsection{Affichage automatique ou manuel}
\label{sec:aam}
Il y a deux modes d'affichage, automatique ou non. Quand l'affichage 
automatique est activé, chaque dessin d'objet s'affiche automatiquement. 
Quand il n'est pas automatique c'est l'appel à la fonction:
\code{
void affiche\_all();
}
qui affiche les objets.

Pour basculer de l'affichage automatique au non automatique, il y a 
deux fonctions:
\code{
void affiche\_auto\_on(); \\
void affiche\_auto\_off();
}
Sur les ordinateurs lents, il vaut mieux ne pas afficher 
les objets un par un, mais les afficher que quand c'est nécessaire.
C'est aussi utile pour avoir un affichage plus fluide quand on fait
des animations.

Par défaut on est en mode automatique.

\subsection{Création de couleur}
La fonction
\code{
COULEUR couleur\_RGB(int r, int g, int b);
}
prend en argument les 3 composantes rouge (\texttt{r}), vert (\texttt{g})
et bleue (\texttt{b}) qui doivent être comprises
dans l'intervalle~$[0..256[$ et renvoie la couleur.


\section{Gestion du clavier et de la souris}
\subsection{Clavier}
La fonction (écrite par Yann Strozecki)
\code{
int get\_key();
}
renvoie le code ascii du caractère de la dernière touche du clavier qui a été pressée.
Si aucune touche n'a été pressée, renvoie -1.
Cette fonction est non bloquante.

Si depuis le dernier appel à la fonction:
\code{
POINT get\_arrow();
}
il y a eu $n_g$ appuis sur la flèche gauche, $n_d$ appuis sur la flèche droite
$n_h$ appuis sur la flèche haut et $n_b$ appuis sur la flèche bas.
Le point renvoyé vaudra en \texttt{x} $=n_d-n_g$ et en \texttt{y} $=n_h-n_b$.

Cette fonction est non bloquante, c'est à dire que si aucune flèche n'a été
appuyée les champs \texttt{x} et \texttt{y} vaudront~0.

\subsection{Déplacements la souris}
La fonction:
\code{
POINT get\_mouse();
}
renvoie le déplacement de souris avec la même sémantique que 
\texttt{POINT get\_arrow();}. Cette fonction est non bloquante:
si la souris n'a pas bougé les champs \texttt{x} et \texttt{y} vaudront 0.

\subsection{Clics de la souris}
La fonction:
\code{
POINT wait\_clic();
}
attend que l'utilisateur clique avec le bouton gauche de la souris
et renvoie les coordonnées du point cliqué. Cette fonction est bloquante.

La fonction:
\code{
POINT wait\_clic\_GMD(char *button);
}
attend que l'utilisateur clique  et renvoie dans \texttt{button} le bouton cliqué:
\texttt{G} pour gauche, \texttt{M} pour milieu et \texttt{D} pour droit.
Cette fonction est bloquante.

La fonction (écrite par Yann Strozecki):
\code{
POINT get\_clic();
}
renvoie la position du dernier clic de souris.
Si le bouton gauche n'a pas été pressée, renvoie $(-1,-1)$.
Cette est fonction non bloquante.

\subsection{Fin de programme}

La fonction:
\code{
POINT wait\_escape();
}
attend que l'on tape la touche \texttt{Echap} ou \texttt{esc} et termine le programme.

\section{Dessin d'objets}
Cette fonction remplit tout l'écran de la couleur en argument:
\code{
void fill\_screen(COULEUR coul);
}

Les fonctions ci-dessous dessinent l'objet dont elle porte le nom.
Quand le nom de la fonction contient \texttt{fill}, l'intérieur est également
colorié.
\code{
void pixel(POINT p, COULEUR coul);\\
void draw\_line(POINT p1, POINT p2, COULEUR coul);

\vspace{3mm}
\noindent
void draw\_rectangle(POINT p1, POINT p2, COULEUR coul); \\
void draw\_fill\_rectangle(POINT p1, POINT p2, COULEUR coul);

\vspace{3mm}
\noindent
void draw\_circle(POINT centre, int rayon, COULEUR coul);\\
void draw\_fill\_circle(POINT centre, int rayon, COULEUR coul);

\vspace{3mm}
\noindent
void draw\_triangle(POINT p1, POINT p2, POINT p3, COULEUR coul);\\
void draw\_fill\_triangle(POINT p1, POINT p2, POINT p3, COULEUR coul);

\vspace{3mm}
\noindent
void draw\_ellipse(POINT f1, POINT f2, int r, COULEUR coul);\\
void draw\_fill\_ellipse(POINT f1, POINT f2, int r, COULEUR coul);
}


\section{Écriture de texte}
Pour l'affichage de texte; il n'y a qu'une seule police: verdana.
La police peut-être changée dans le fichier \texttt{graphics.h},
mais nécessite ensuite une recompilation de la librairie et une reinstallation.

Pour les lettres accentuées, il faut respecter l'encodage ttf.
Pour ce faire on peut utiliser les contantes définies (voir annexe).
Par exemple:
\code{
char *s = "Cha"icirc"ne de caract"egrave"re";
}
\texttt{icirc} vaut la chaîne \texttt{"\textbackslash xee"} qui est \texttt{î} et on utilise la concaténation
de constantes de type chaînes de caractère.

L'écriture se fait avec la fonction:
\code{
void aff\_pol(char *a\_ecrire, int taille, POINT hg, COULEUR coul);
}
\begin{itemize}
\item \texttt{a\_ecrire} est la chaîne de caractère contenant le texte à écrire;
\item \texttt{taille} est la taille de la police;
\item \texttt{hg} fourni les coordonnées en haut à gauche où sera positionnée le texte;
\item \texttt{coul} est la couleur d'écriture.
\end{itemize}

La même chose mais en donnant le point central du texte et non plus
le point en haut à hauche:
\code{
void aff\_pol\_centre(char *a\_ecrire, int taille, POINT centre, COULEUR coul);
}

Pour positionner comme on le souhaite un texte, les deux fonctions:
\code{
int largeur\_texte(char *a\_ecrire, int taille);\\
int hauteur\_texte(char *a\_ecrire, int taille);
}
renvoie la largeur et la hauteur, en pixels, du texte à écrire.

Le style peut être changé grâce à la fonction
\code{
void pol\_style(int style);
}
Les styles possibles sont : \texttt{NORMAL}, \texttt{GRAS}, \texttt{ITALIQUE} et 
\texttt{SOULIGNE}. Le style est appliqué jusqu'au prochain appel de la fonction.
Au départ le style est \texttt{NORMAL}.

La fonction ci-dessous affiche directement un entier sans avoir 
à le mettre dans une chaîne de caractères:
\code{
void aff\_int(int n, int taille, POINT p, COULEUR C);
}

Les fonctions suivantes affichent dans la fenêtre graphique comme dans 
une fenêtre shell. Commence en haut et se termine en bas mais il n'y a
pas de scroll. La police est de taille 20 et de couleur blanc.

\'Ecriture d'une chaîne de caractères, d'un entier, d'un flottant ou d'un booléen:
\code{
void write\_text(char *a\_ecrire);\\
void write\_int(int n);\\
void write\_float(int n);\\
void write\_bool(BOOL b);
}

La fonction:
\code{
void writeln();
}
effectue un retour à la ligne.

\section{Lecture d'entier}
Renvoie d'un entier tapé au clavier. 
\code{
int lire\_entier\_clavier();
}
Cette fonction est bloquante


\section{Gestion du temps}

\subsection{Chronomètre}
Ce chronomètre est précis à la microseconde.
Il y a deux fonctions, l'une démarre le chrono:
\code{
void chrono\_start();
}
De plus, elle remet le chrono à zéro s'il a déjà été démarré.
La fonction:
\code{
float chrono\_lap();
}
renvoie la valeur courante du chrono sans l'arrêter.
Elle peut donc être appelée autant de fois que nécéssaire.

\subsection{Attendre}
Attend le nombre de millisecondes passé en argument:
\code{
void attendre(int millisecondes);
}

\subsection{L'heure}
Renvoie la valeur de l'heure courante:
\code{
int heure();
}

Renvoie la valeur de l'heure courante:
\code{
int minute();
}

Renvoie le nombre de secondes à la microseconde près:
\code{
float seconde();
}

\section{Valeur aléatoires}
Renvoie d'un \texttt{float} uniformément distribué dans l'intervalle $[0;1[$:
\code{
float alea\_float();
}

Renvoie d'un \texttt{int} uniformément distribué dans l'intervalle $[0..N[$
soit $N$ valeurs différentes de 0 à $N-1$:
\code{
int alea\_int(int N);
}

\section{Divers}
Calcule de la distance entre deux points:
\code{
float distance(POINT P1, POINT P2);
}

\section{Installation}
L'installation se fait dans une fenêtre shell sur une machine linux ubuntu.
\begin{enumerate}
\item Récupérer le fichier \texttt{UVSQ\_graphics.zip}
\item Décompresser et aller dans le dossier \texttt{UVSQ\_graphics}
\code{
\$> unzip UVSQ\_graphics.zip\\
\$> cd UVSQ\_graphics
}
\item Pour installer, taper:
\code{
\$> make install
}
\item Pour compiler et exécuter un programme, par exemple \texttt{demo1.c}, taper:
\code{
\$> makeuvsq demo1\\
\$> ./demo1
}
\end{enumerate}

\section{Un exemple complet}
Le fichier \texttt{demo1.c}:
\code{
\verbatiminput{demo1.c}
}

\newcommand\letac[2]{\#define #1 "\textbackslash #2"\\}
\newcommand\letacfin[2]{\#define #1 "\textbackslash #2"}

\appendix
\section{Annexes}
\subsection{Liste des caractères accentués}
La dénomination est identique à celle utilisée en html.
\code{
\begin{tabular}{|c|l|}
\hline
à   & \letac{agrave}{xe0}
\'a & \letac{aacute}{xe1}
â   & \letac{acirc~}{xe2}
ä   & \letac{auml~~}{xe4}
ç   & \letac{ccedil}{xe7}
è   & \letac{egrave}{xe8}
é   & \letac{eacute}{xe9}
ê   & \letac{ecirc~}{xea}
ë   & \letac{euml~~}{xeb}
î   & \letac{icirc~}{xee}
ï   & \letac{iuml~~}{xef}
ô   & \letac{ocirc~}{xf4}
ù   & \letac{ugrave}{xf9}
û   & \letac{ucirc~}{xfb}
ü   & \letac{uuml~~}{xfc}
\c{C} & \letac{Ccedil}{xc7}
È   & \letac{Egrave}{xc8}
\'E & \letac{Eacute}{xc9}
Ê   & \letac{Ecirc~}{xca}
\hline
\end{tabular}
}

\subsection{Liste des codes de couleur en français}
\code{
\#define~argent~~~~~0xC0C0C0\\
\#define~blanc~~~~~~0xFFFFFF\\
\#define~bleu~~~~~~~0x0000FF\\
\#define~bleumarine~0x000080\\
\#define~citronvert~0x00FF00\\
\#define~cyan~~~~~~~0x00FFFF\\
\#define~magenta~~~~0xFF00FF\\
\#define~gris~~~~~~~0x808080\\
\#define~jaune~~~~~~0xFFFF00\\
\#define~marron~~~~~0x800000\\
\#define~noir~~~~~~~0x000000\\
\#define~rouge~~~~~~0xFF0000\\
\#define~sarcelle~~~0x008080\\
\#define~vert~~~~~~~0x00FF00\\
\#define~vertclair~~0x008000\\
\#define~vertolive~~0x808000\\
\#define~violet~~~~~0x800080
}

\subsection{Liste des noms de couleur en français et en anglais}
\includegraphics[height=0.8\textheight]{les_couleurs.png}

\url{http://fr.wikipedia.org/wiki/Liste_de_couleurs}
\end{document}
